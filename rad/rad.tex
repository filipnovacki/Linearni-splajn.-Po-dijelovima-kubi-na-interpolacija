\documentclass[12pt,a4paper]{report}
\usepackage[croatian]{babel}
\usepackage[utf8]{inputenc}
\usepackage{amsmath}
\usepackage[margin=3cm]{geometry}
\usepackage{amsfonts}
\usepackage{amssymb}
\usepackage{graphicx}
\usepackage{url}
\usepackage{wrapfig}

\author{Matea Novak \and Filip Novački \and Ante Vulić}
\title{Linearni splajn. Po dijelovima kubična interpolacija}
\begin{document}
	\maketitle
	
	\tableofcontents
	
\chapter{Uvod}
	Ovo je timski projektni zadatak napravljen je u sklopu kolegija Odabrana poglavlja matematike na Fakultetu organizacije i informatike u Varaždinu Sveučilišta u Zagrebu. Cijeli projekt razvijan je na \texttt{git} repozitoriju koji je dostupan na: \url{https://github.com/filipnovacki/linearni-splajn}. 
\chapter{Zadaci}
	\section{Opis glavne ideje po dijelovima polinomne interpolacije}
	Kod raznih mjerenja, istraživanja i sl. se događa da rezultati nekog testiranja budu podaci koji su diskretni, odnosno izmjereni su za neke vrijednosti. Kako bi nam te vrijednosti bile korisne i primjenjive, moramo na neki način odrediti koje su vrijednosti između dviju diskretnih vrijednosti. Tu u priču dolazi interpolacija.
	
	\begin{wrapfigure}{r}{0.55\textwidth}
		\includegraphics[width=0.55\textwidth]{slike/usporedba.png}
		\caption{Usporedba linearne interpolacije na isječku funkcije $f(x)=\sin x$. Plavom bojom je prikazana interpolacija, a narančastom funkcija. Izvor: autorska izrada}
		\label{usporedba}
	\end{wrapfigure}
	
	Jedna je opcija napraviti interpolaciju linijama prvog reda, dakle, pravcima, odnosno dužinama koje jednostavno povučemo između dviju točaka. Takva aproksimacija je vrlo jednostavna za izračunati, međutim, takvo mjerenje ne daje baš zadovoljavajuće rezultate. Drugi je problem što mjesta gdje se ti pravci spajaju su \textit{oštri}, odnosno ne postoji glatki prijelaz.
	
	Kao što vidimo na slici \ref{usporedba}, plava linija dosta je dosta blizu narančastoj, no za razne primjene bi ta greška mogla biti prevelika.
	
	Bolja opcija je povezati točke krivuljama višeg reda. Kako bismo dobili glatki prijelaz, moramo paziti da derivacija u točkama uvijek bude jednaka. Kvadratični splajn ima jedan nedostatak. Nedostatak je u tome što je zakrivljen uvijek u jednu stranu, odnosno u obliku parabole. Kubična interpolacija je po tom pitanju mnogo bolja jer može imati oblik parabole, ali ne mora. 
	
	Mogu se koristiti i polinomi još višeg reda, međutim oni se nešto teže računaju te funkcije polinoma višeg reda imaju tendenciju na nekim mjestima jako \textit{divljati}, što ponekad može predstavljati problem.
	\section{Opis linearnog splajna}
	
	\section{Primjeri}
		\subsection{Traženje algoritma za traženje linearnog splajna}
		\subsection{$f(x)=\sin x$}
		\subsection{$g(x)=\frac{1}{x^2 +1}$}
	\section{Opis nedostataka linearnih splajnova}
	\section{Opis po dijelovima kubične interpolacije}
	\section{Primjeri}
		\subsection{Traženje algoritma za traženje po dijelovima kubične interpolacije}
		\subsection{$f(x)=\sin x$}
		\subsection{$g(x)=\frac{1}{x^2 +1}$}
	\section{Zaključak}
\end{document}